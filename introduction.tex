\chapter{Introduction}
\label{sec:introduction}

Ceph is a highly distributed, strongly consistent file system. It
stores data redundantly within a data center using a type of hashing
algorithm. This hashing algorithm determines how data is apportioned
to storage nodes.  By calculating hash information independently, a
Ceph client may communicate directly with the correct storage node
without having to consult a separate controller node. The result of
this approach is a highly distributed control structure, free of
single points of failure, which allows Ceph clusters to be very
robust. This communications structure also has the benefit of
improving overall performance and scalability by spreading control and
communication across all nodes.

For this project, we were tasked with determining a means in which to
asynchronously geo-replicate the contents of a Ceph cluster. The task
also specifies requirements for performance and consistency. We were
able to find a solution to the problem presented to us, and our
completed goals are in line with those that we set at the beginning of
the project (TODO reference Appendix I). In this document, we are
including the theoretical proof that this solution is correct and
performant (TODO reference Proof), as well as testing that shows the
same thing (TODO reference Analysis). In addition to our promised
deliverables, we are also delivering a set of utilities and scripts to
replicate and extend our analysis.
