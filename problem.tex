\chapter{Description of Problem}
\label{sec:description}

As Ceph nodes must store data in several replicas, synchronous data
replication within a data center is fundamental to the operation of
Ceph. Ceph currently does not implement a manner in which to
asynchronously replicate a total data set across geographically
separated data centers. The design of such an asynchronous replication
feature is the goal of this project.  This feature would enable usage
scenarios such as data center fail-over. One of the largest challenges
in implementing this feature is obtaining a consistent snapshot of the
total data set at a given point in time. A consistent snapshot here
means a partition of the set of events (reads and writes) in the
system such there there is no event excluded from the snapshot that an
event included in the snapshot depends on. This consistency is
necessary from the point of view of a client to the file system.

As Ceph is a distributed file system, without controller nodes that
have a full picture of the system, there are inherent challenges to
acquiring a complete picture of the system at any given moment in
time. Ceph is also strongly consistent, so a potential solution is
further complicated as all data that a particular piece of data relies
on must also be available. For this problem, a snapshot is allowed to
be slightly out of date, but it must present a view of the file system
that the file system could have been in at some point in the past. To
have older but consistent data is preferable to newer but inconsistent
data. This consistency must hold from the perspective of a
client. This means that the system must not assume that it has full
knowledge of event dependencies, and the system must take into account
the possibility of out-of-band communication between clients. For
example, there could be multiple users or hosts using the same Ceph
distributed file system. These clients could be communicating
out-of-band with each other (and, as a result, writing to the file
system) without the file system’s knowledge. In this case, the file
system would not know that these events depend on each other, yet we
would still need to ensure that our snapshots are consistent.

The Ceph architecture is designed to grow to hyper-scale. As a result,
scalability and performance are of primary concern. The time necessary
to snapshot in any proposed solution must not severely increase as
more clients and nodes are added. In some production scenarios,
snapshots must occur with a minimum frequency regardless of cluster
size. Similarly, snapshots must have negligible impact on users. For
example, one way to provide a consistent snapshot is to simply block
all reads and writes to the Ceph file system while the state of the
system is gathered. However, this solution would have a substantial
impact on input/output operations, making this solution infeasible.

At the scale of a Ceph deployment, node failures are routine. Any
solution must take this into account and treat failures as the rule,
not the exception. A solution must consider the impact of node
failures on the consistency of a snapshot and be able to adapt
accordingly. Similarly, a potential solution must assume that
components on which it relies can fail at any time.
