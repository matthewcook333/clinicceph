\chapter{Approach}
\label{sec:approach}

We have developed a simple, performant clock-synchronization-based
algorithm that can provide consistent data center level snapshots.

The algorithm has a few requirements. First, it requires that the
clock synchronization algorithm used by the data center support error
bounding. The most common NTP implementation, ntpd, supports
this. Second, to remain performant it requires a certain baseline
clock and network link quality, although it will degrade gracefully
with poor clocks or links (TODO reference results section)

The algorithm we propose has 4 phases. They are as follows:

\begin{enumerate}

\item \emph{Synchronization}

  This “phase” is happening continuously in the data center. All nodes
  sync their clocks to a common master clock using a synchronization
  algorithm supporting error bounds as discussed above. Additionally,
  snapshot times are pushed out (possibly via map updates or embedding
  in heartbeat replies).

\item \emph{Freeze}
  
  Nodes hold incoming writes. The writes may be processed, but
  completion is not acknowledged. Let $U_i$ be the uncertainty in node
  $i$’s current time, and $T$ be the scheduled snapshot time. Node $i$
  begins its freeze when its clock reads $T - U_i$, and completes it
  when its clock reads $T + U_i$, guaranteeing that the master clock’s
  $T$ is captured in the freeze window for all $i$. (TODO reference
  proof section)

\item \emph{Confirmation}

  Before a snapshot may be marked as good, the data center must wait a
  short period. This allows any sudden clock desynchronization events
  or node failures to be detected, and consistency
  checked. Specifically, RADOS verifies that for each failure, at
  least one other member of all the PGs on that node did not fail or
  have clock desynchronization events. If no errors are found, the
  snapshot is marked as good

\item \emph{Replication}
  
  Finally, if the snapshot was marked as good after phase 3, the data
  may be replicated. Because the first step in taking a snapshot is
  simply writing down a marker in an OSD’s log, it is straightforward
  either to do a full snapshot, or just a diff since the last good
  snapshot.

\end{enumerate}
