\chapter{Introduction}
\label{sec:introduction}

Ceph is a highly distributed, strongly consistent file system. It
stores data redundantly within a data center using a type of hashing
algorithm. This hashing algorithm determines how data is divided among
storage nodes.  By calculating hash information independently, a Ceph
client may communicate directly with the correct storage nodes rather
than relying on a separate controller node. The result of this
approach is a distributed control structure, free of single points of
failure. This allows Ceph clusters to be very robust. This
communication structure also has the benefit of improving overall
performance and scalability by spreading control and communication
load across all nodes.

For this project, we were tasked with determining a means by which to
asynchronously geo-replicate the contents of a Ceph cluster. The task
also specifies requirements for performance and consistency. We were
able to find a solution to the problem presented to us, and our
completed goals are in line with those that we set at the beginning of
the project (see appendix~\ref{sec:plan} for details).
Chapter~\ref{sec:proof} demonstrates both the theoretical correctness
and theoretical performance of our
algorithm. Chapter~\ref{sec:analysis} covers a concrete demonstration
of the same. In addition to our promised deliverables, we are also
delivering a set of utilities and scripts to replicate and extend our
analysis for Ceph clusters to which snapshotting is being rolled out.
