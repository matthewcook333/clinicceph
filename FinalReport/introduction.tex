\chapter{Introduction}
\label{sec:introduction}

Ceph is a highly distributed, redundant, strongly consistent data
center-level file system. It is robust to failures within a data
center, but does not currently provide a mechanism to protect against
a data center-wide failure (e.g. a severed network fiber line).

A Ceph client is able to communicate directly with the storage nodes
containing data it accesses rather than relying on a separate
controller node. This gives a distributed control structure, free of
single points of failure. This allows Ceph clusters to be very
robust. This communication structure also has the benefit of improving
overall performance and scalability by spreading control and
communication load across all nodes.

For this project, we were tasked with determining a means by which to
asynchronously geo-replicate the contents of a Ceph cluster. The data is 
transfered in recordings of the global state of the cluster, called snapshots.
These snapshots must be consistent and performant, which required overcoming 
the following complications:

\begin{itemize}
\item Inability to perfectly synchronize clocks between machines.
\item Lack of knowledge of causality relationships between reads and
  writes in the cluster, because Ceph permits out-of-band
  communication.
\item Insufficient per-node knowledge to fully reconstruct the state
  of a cluster at any specific time as Ceph is currently implemented.
\end{itemize}

Our snapshotting algorithm, called the Consistent Asynchronous Snapshotting via Time Syncrhonization (CASTS), was to use a time synchronization protocol that provides
uncertainty bounds on the error for each node's clock in the cluster. These
uncertainty bounds can be used to create brief periods at each node where writes are 
held, called freeze windows. By relating the freeze windows to the uncertainty 
in the clock error, we can be certain that they overlap, and that overlap
is when we can take a consistent snapshot of the system. More details
about CASTS are covered in Chapter~\ref{sec:approach}.

There are a number of other solutions to problems similar to
ours. Logical clocks are often used to order events, as well as
timestamping. Google has implemented a library called TrueTime, which
provides similar uncertainty bounds on the real time in their Spanner
database. Chapter~\ref{sec:rel-work} describes these algorithms and
why they are not satisfactory solutions for our problem.

% chapter overview paragraph
Chapter~\ref{sec:proof} covers a theoretical proof for CASTS,
arguing that our solution will allow us to take a consistent snapshot
by preserving causal relationships of events. Chapter~\ref{sec:proof} also
contains a performance analysis of our proof, arguing that the freeze
windows will not be long enough to impact Ceph's I/O availability and that our
solution will not put a significant burden on network availability or
on the nodes in the cluster.

Chapter~\ref{sec:results} covers our concrete testing of our solution,
which includes:

\begin{itemize}
\item the collection of data from a Ceph cluster to create
  latency and clock drift models;
\item simulations showing that the uncertainty bounds defined by the
  Network Time Protocol (NTP) will contain the real time;
\item simulations that test NTP's performance in a multitude of
  cluster configurations.
\end{itemize}
The goal of Chapter~\ref{sec:results} is to show that our solution will work
in practice, not just in theory.

Chapter~\ref{sec:impl} contains implementation details and
recommendations for a team that will be implementing CASTS
into Ceph. Chapter~\ref{sec:future} discusses work that could be done
to improve on our solution.

In addition to this report, we are also delivering a set of scripts to
replicate and build on our analysis. These scripts will also be useful
in analyzing clusters on which consistent geo-replication is
enabled. Using the results of this clinic project, a future team
should be able to quickly implement CASTS in Ceph.

In Appendix~\ref{sec:terms} we provide
a glossary including technical terms we use in this report.
