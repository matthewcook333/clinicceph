\chapter{Implementation Details}
\label{sec:impl}

It remains necessary that \alg be implemented in Ceph. 
A future team, when implementing \alg, should keep in 
mind a few considerations. In particular, a team should 
remember that a time server and client cannot be 
synchronized instantaneously, that the configuration 
of the NTP server strongly influences the size of the 
uncertainty window, and that NTP may not be an 
appropriate synchronization protocol for all network topologies. 

NTP requires time to stabilize its estimate of a clock's drift.
Clock drift rate factors into NTP's reported maximum uncertainty
term. This means NTP implementations will not report a 
valid maximum uncertainty value
until after they have a stable estimate of clock drift.
As a valid maximum uncertainty term is necessary for the proper 
functioning of \alg, a newly online machine may not take part in
a snapshot immediately. It must wait until NTP begins reporting 
valid uncertainty values, a period of a few thousand seconds after 
time synchronization begins. Newly online machines then have a
``warm up'' period in which a snapshot cannot trust their freeze
window. During implementation, a team may decide to prevent 
a new node from being previsioned with data until after this period 
is expired, or the team may decide to allow the node to 
become provisioned immediately and simply rely on the node's 
replicas to ensure data consistency in snapshots. Once 
the machine's NTP implementation begins reporting
valid uncertainty values, the machine can be fully
incorporated into the Ceph cluster.

If the primary NTP server were to fail, then all nodes in the
cluster would have to fail over to a secondary server.
This would result in another period of computing estimates. 
Ceph would therefore
have to require a short suspension of snapshotting until all clocks in
the data center re-calculate their estimates.

The quality of the master NTP clock significantly impacts the size of
NTP's uncertainty bounds. If specialized hardware is a possibility, we 
suggest that a very accurate clock, like
a GPS or atomic clock, be used in the Ceph cluster. This
recommendation is an implication of how NTP synchronizes clocks in a
network. NTP specifies a hierarchy of clocks, where clocks in lower strata
in the hierarchy synchronize themselves with clocks that are in the
strata above them \citep{Burbank2010}. Clocks that are higher in the 
hierarchy (have lower strata) are
generally more accurate than clocks lower in the hierarchy. At the top
of the hierarchy are stratum 0 clocks, which are very accurate and are
used as the reference for real time. Stratum 0 clocks are generally 
devices like GPS clocks or atomic clocks. By including a very accurate
clock in the cluster, like a GPS or atomic clock, nodes in the data center
will not have to consider uncertainty introduced by multiple layers 
of time servers, nor from communications that reach outside of the
data center. Without a stratum 0 device directly connected to the 
data center's local NTP server, the NTP implementation is forced to synchronize
to a remote time server, outside of the data center, adding a significant
amount of uncertainty. Without a stratum 0 device, a local NTP server
may also use its on-board RTC clock without synchronization. However, 
NTP implementations then assume to worst possible clock
properties for the server's physical RTC clock, also significantly 
increasing uncertainty. Without a local stratum 0 device, \alg will
still function correctly. It will simply require larger, more noticeable,
freeze windows. 

NTP functions well as a time synchronization protocol for \alg.  
However, another time synchronization protocol, more
tailored to properties of data center networks 
(high speed, low latency, managed networks), 
could provide tighter uncertainty
bounds than NTP. Our analysis here and in Chapter~\ref{sec:results} 
focuses on the NTP implementation \texttt{ntpd} as \texttt{ntpd} is widely
deployed, it supports hard bounds its own uncertainty, and it 
provides easy access to important
diagnostic information. However, if a better time synchronization
protocol with tighter uncertainty bounds could be found, it should be
used to achieve shorter freeze windows. The Precision Time Protocol (PTP)
and the NTP implementation Chrony are each candidates for use with \alg. 
They each may need modification to extract the
necessary uncertainty information however.
