\chapter{Implementation Details}
\label{sec:impl}

\alg has been developed Our project is to develop an algorithm that would be implemented by a
future team \alg. \alg In this chapter, we discuss the issues that an
implementation team should consider when incorporating our algorithm
into Ceph.

NTP requires time to stablize its estimation of the clock's drift.
NTP not only uses the last-seen error but also an estimated drift
rate, which requires past data points to calculate. This calculation
is required for each node in the cluster, so when a new machine is
added to the cluster, it could not be used in snapshots until roughly
a few thousand seconds after it was added to the cluster. Before that
point, the node's clock offset could not be estimated and therefore,
we could not guarantee that the node's data in a snapshot would be
consistent. This could be resolved by requiring new machines to have a
``warm up" period before they are actually assigned data. Once the
machine's first NTP estimate is calculated, it can be fully
incorporated into the Ceph cluster.

If the primary NTP master clock were to fail, then all nodes in the
cluster would have to recompute the estimate of the clock drift,
which, as mentioned above, requires time to do. Ceph would, therefore,
have to require a short suspension of snapshotting until all clocks in
the data center re-calculated their estimates.

The quality of the master NTP clock significantly impacts the size of
NTP's uncertainty bounds. We suggest that a very accurate clock, like
a GPS or atomic clock, be added to the Ceph cluster. This
recommendation comes from how NTP synchronizes clocks in a
network. NTP has a hierarchy of clocks, where clocks in lower strata
in the hierarchy synchronize themselves with clocks that are in the
strata above them. Clocks that are higher in the hierarchy are
generally more accurate than clocks lower in the hierarchy. At the top
of the hierarchy are stratum 0 clocks, which are very accurate and are
used as the reference for real time. By including a very accurate
clock in the cluster, like a GPS or atomic clock, the master clock in
the NTP server can synchronize with that clock, allowing it to move up
to stratum 1. NTP can then eliminate upstream variation when
calculating the uncertainty, which reduces the uncertainty bounds it
creates. Without a stratum 0 device directly connected to the NTP
stratum 1 server, NTP is forced to assume to worst possible clock
properties for the on-board RTC. While this recommendation is not
strictly necessary, a stratum 1 clock will provide shorter freeze
windows in our algorithm.

NTP functions well as a time synchronization protocol for our
algorithm.  However, another time synchronization protocol, more
tailored to data center precision, could provide tighter uncertainty
bounds than NTP. NTP was used for our analysis because it is widely
deployed, it can bound its own uncertainty, and the reference
implementation of the protocol provides easy access to important
diagnostic information. However, if a better time synchronization
protocol with tighter uncertainty bounds could be found, it should be
used to achieve shorter freeze windows. The Precision Time Protocol
and Chrony (another NTP implementation), are possible candidates for
investigation, though they may need modification to extract the
necessary information.
