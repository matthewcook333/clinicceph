\chapter{Conclusion}
\label{sec:conclusion}

The goal of our project was to design a snapshotting system for Ceph 
that allows for asynchronous replication of data to remote data centers.
We developed an algorithm that allows for taking these consistent, 
point-in-time snapshots by synchronizing clocks within a Ceph cluster
and performing short write holds for a length of time directly 
proportional to the uncertainty of the clock synchronization accuracy.
Even in the presence of out-of-band communication, the algorithm provides
consistency among all of the events even without external dependency
information. Our algorithm runs at the level of each individual node,
which means that a Ceph data center with this algorithm remains highly
scalable without increasing impact on the system. Also, with
reasonable guarantees on the uncertainty bounds provided by NTP, the
freeze windows, and as a result the snapshot, should have negligible
impact on a Ceph user. The algorithm is also robust to node failures
since as long as one replica of each node remains up, the snapshot can be
taken and the algorithm can succeed. We have found the NTP to be
satisfactory for the usage in our algorithm in fulfilling the problem
constraints, though we have also outlined alternatives and possible
future work for using and testing other time synchronization
protocols.
