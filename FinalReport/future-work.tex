\chapter{Future Work}
\label{sec:future}

With our research and analysis, we have also found points of interest
should Red Hat or a future clinic team choose to pursue them. For one,
\alg can be used with any time synchronization
protocol that provides guarantees on the clock uncertainty bounds. As
such, alternative time protocols such as Chrony or the Precision Time
Protocol (PTP) can be investigated to see if they could provide
tighter uncertainty bounds than NTP. As of the time of this writing,
Chrony and PTP implementations cannot provide guarantees on the
uncertainty bounds, but could possibly be extended to include them and
perhaps improve upon the bounds of \texttt{ntpd}.

In addition, \alg could be tested on a wider range
of data center, network, and server configurations.
The testing we performed is more representative of smaller data
centers with a single NTP server for synchronization. Other tests
could include very large data centers, which could result in more uncertainty 
in observed network latency values and, more importantly, could require 
more than one NTP server (and thereby increase uncertainties introduced by
multiple layers of synchronizing clocks Chapter~\ref{sec:impl}).

For our simulations, the model we currently use for clock drifting
behavior is a normal distribution. Unfortunately, we were
unable to find literature describing better models for clock drifting
behavior. With further analysis and testing, a better model could be
found for clock drifting behavior to help analyze the performance of
the behavior. The clock drifting behavior is dependent on
variables such as the Quartz crystal used in the clock, temperature and pressure
variance in the room, and changes due to the synchronization process.



Lastly, \alg still needs to be implemented in 
Ceph. We outlined particular considerations for implementation
in Chapter~\ref{sec:impl}. We believe that with a comfortable
understanding of the Ceph infrastructure, \alg will be straightforward 
to implement in Ceph.
