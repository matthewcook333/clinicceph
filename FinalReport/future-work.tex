\chapter{Future Work}
\label{sec:future}

With our research and analysis, we have also found points of interests
should Red Hat or a future clinic team choose to pursue them. For one,
the algorithm proposed can be used with any time synchronization
protocol that provides guarantees on the clock uncertainty bounds. As
such, alternative time protocols such as Chrony or the Precision Time
Protocol (PTP) can be investigated to see if they could provide
tighter uncertainty bounds than NTP. As of the time of this writing,
those two particular protocols cannot provide guarantees on the
uncertainty bounds, but could possibly be extended to include it and
perhaps improve upon the bounds of NTP.

In addition, the algorithm proposed could be tested on a wider range
of scenarios than was done for this project to prove its
performance. The testing we did is more representative of smaller data
centers with a single NTP server for synchronization. Other tests
could include very large data centers, which would give more
uncertainty on the network latency values and, more importantly, use
more than one NTP server.

For our simulations, the model we currently use for clock drifting
behavior is a consistently normal distribution. Unfortunately, we were
unable to find literature describing better models for clock drifting
behavior. With further analysis and testing, a better model could be
found for clock drifting behavior to help analyze the performance of
the behavior. The clock drifting behavior is very dependent on
variables such as the Quartz crystal used, temperature and pressure
variance, and changes due to the synchronization process.



Lastly, our algorithm still needs to be implemented and incorporated
into Ceph. Our recommendations to approach this challenge are outlined
in section~\ref{sec:impl}. We believe that with a comfortable
understanding of the Ceph infrastructure, our algorithm could be
implemented without significant struggle.
