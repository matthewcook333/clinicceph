\chapter{Related Work}
\label{sec:rel-work}

Creating a consistent snapshot of a Ceph cluster is difficult because
Ceph is distributed: The system does not have a single or small group
of nodes that can track every I/O operation. Instead, the cluster must
aggregate knowledge from potentially thousands of nodes to capture a
complete point-in-time snapshot of its data.  Significant research
efforts have been directed toward taking snapshots as well as
synchronizing and ordering events in distributed
systems. Unfortunately, none of these approaches can address the
particular constraints of Ceph outlined in Chapter
\ref{sec:description} without modification.

\section{Logical Clocks}

Lamport clocks use a ``logical clock'' to establish a partial ordering
of events in a distributed system. If an ordering of the events and
messages between nodes in a distributed system can be established, and
this ordering is complete enough to take into account all
dependencies, a snapshot may be obtained by observing which events
happen before other events~\citep{Lamport1978}.

Lamport clocks try to order events and messages by a logical
timestamp, a counter that increments following each event or message
pass, rather than by some physical time that could be affected by
clock drift. This ordering method works well if the system can reason
about events that are occurring between its own nodes. However, this
type of partial ordering fails to guarantee correct ordering in the
case of out-of-band communication as described in Chapter
\ref{sec:description}; the system cannot order the events that occur
outside of the system as it has no knowledge of the order of those
events. These clocks might still be useful for reasoning about
interactions that occur strictly within the Ceph cluster, but they
require a second system that supports out-of-band communication to
meet Ceph's consistency guarantee.

Vector clocks are very similar to Lamport clocks. Instead of
maintaining a single logical time, each node stores a vector of
times. Each element of the vector represents nodes in the
system.  The value of each element is the time of the last event
received by that node in the system. The ordering of events is
based on the relative times stored in the vectors. All messages
contain the sending node's vector and a node updates its vectors when
it receives a message. Vector clocks can be superior to Lamport clocks
because they do not assign an arbitrary counter
representing time to events. Instead, they use the relative times of
other nodes to determine the partial order of the events. However,
like Lamport clocks, out-of-band communication is not accounted for
and events could be incorrectly ordered if we tried to use vector clocks for 
our snapshotting algorithm~\citep{Fidge1988}.

\section{Network Time Synchronization}

As mentioned in Chapter \ref{sec:description}, total ordering can be
achieved through timestamps if we used the timestamp of each event to 
determine to chronologically order the events. 
In theory, this would be able to provide
a total ordering of events including any out-of-band
communication. Clock drift and disagreement, or ``skew'', complicate this approach: When
multiple clocks are in use in a distributed system, the clocks must be
synchronized in order to use timestamps to order events across nodes.

Time synchronization protocols like the Network Time Protocol (NTP)
could help keep the clocks in the system from desynchronizing too
greatly. In theory, a time synchronization protocol could be so effective that 
the clock errors in the system would be small enough to allow for timestamps 
alone to be used to order events. However, no time synchronization protocol 
we have been able to
find can maintain clock synchronization well enough that would allow
us to use timestamps alone to order events consistently. 

A synchronization algorithm could be effective should it provide
sufficiently tight, provable bounds on the drift of a given clock and
on the skew across the distributed system. These bounds provide a
method to obtain a consistent snapshot as we described in CASTS
in Chapter~\ref{sec:approach}.

Many clock synchronization algorithms exist. However, these
algorithms tend to have major shortcomings when they are considered
for snapshotting within a Ceph cluster. A
synchronization method must have bounds on the clock error to order
events in a snapshot, and these error bounds must be a hard upper
bound in order for Ceph to make guarantees about the consistency of
its snapshots as described in Chapter~\ref{sec:approach}. The following subsections describe existing time
synchronization methods and protocols.

\subsection{Network Time Protocol (NTP)}

NTP is a robust algorithm for time
synchronization~\citep{Burbank2010}. NTP is designed to synchronize
time at geographically separated computers over the Internet. As a
result, it is resilient to node failure, network unreliability, and
poor clock quality. NTP requires little information about the
network configuration and nodes on which it is operating to provide 
synchronization. NTP uses statistical estimators to predict
future clock performance. Notably, NTP defines a maximum error
term of clock skew in relation to a single root time source. Given this maximum
error term, NTP could be used as a time synchronization method
in CASTS.

However, NTP's focus on robustness
causes compromises for the length of the freeze windows. NTP consistently overestimates
uncertainty (see Chapter~\ref{sec:results} for an in-depth examination
of this issue). A protocol designed for local area network
synchronization could likely do away with some of the complexity of
NTP, make more assumptions about network configuration, and as a
result provide tighter bounds on a node's clock error.

In Chapter~\ref{sec:results}, we analyze the performance of an NTP
implementation, \texttt{ntpd}, across various clock and network
conditions. The \texttt{ntpd} daemon was chosen for its common use and
ease of access to relevant calculated parameters. \texttt{Chrony} is
another potential choice for an NTP implementation, although it lacks
a method to extract the hard error bounds.

\subsection{Precision Time Protocol (PTP)}

PTP is designed for tightly synchronizing
computers on a local network~\citep{2008}. To gain even better
synchronization with tighter bounds on clock uncertainty than NTP, an 
user may use specific hardware that supports PTP. Hardware that
supports PTP is generally included in current data center switches. 
The type of hardware that supports PTP helps decrease the
amount of random variation in message latency in a network, enabling
more accurate measurements of time.  These measurements should 
provide tighter uncertainty bounds. However, the
protocol does not specify (and the implementation does not include) an
upper bound error value. This means that this protocol is not
currently suited for use with CASTS.

If NTP is found to give clock uncertainty bounds that are too large to 
maintain Ceph's performance, it would be worth considering extending PTP to
give an upper bound uncertainty value (Chapter~\ref{sec:impl} has a
brief discussion of this). The implementation of a maximum error term
would be sufficient for PTP to be used with CASTS.

% NOTE: Don't think these are needed. Awkward to add implementation details 
%     into related work
%Linux PTP is an implementation of the Precision Time Protocol 
%(PTP)~\citeyearpar{2008}. PTPd is another
%implementation~\citeyearpar{2008}.

\subsection{Wireless Synchronization}

Wireless time synchronization is an easier problem than wired time 
synchronization. Wireless signal propagation times are easy to model and as a
result time synchronization is straightforward to perform with a high level of
accuracy. Protocols such as PulseSync take advantage of this
observation to achieve highly synchronized clocks~\citep{Lenzen2010}.

GPS clocks, by communicating with satellites to compute the time, can act as 
xnearly perfect time sources. It is
advisable to incorporate a GPS time source (or similarly accurate time
source) into the master clock of any implementation of CASTS
in order to achieve tighter uncertainty bounds, though adding this
would require specialized hardware.  The details of this are discussed
in Chapter~\ref{sec:impl}.

\section{TrueTime}

Google has a number of time-sensitive applications, such as the
synchronously geo-replicated Spanner database. Synchronous
geo-replication of database transactions requires very strong time
guarantees because database transaction ordering is important for
reliability and security.

A library called TrueTime was developed to support these applications.
TrueTime does not provide current time in the way that standard time
libraries do, such as for NTP or PTP. Instead, the ``current time'' it
gives is a range that is guaranteed to contain the current
time. The bounds are claimed to be less than 10
milliseconds. This range in the current time allows for rapid
throughput on the database, since a freeze for the range would only be in effect for 10 milliseconds, while still maintaining definitive
transaction ordering. When TrueTime's guaranteed bounds start to
spread, it will throttle writes to maintain that
ordering~\citep{Corbett2012}.

TrueTime relies on having accurate and precise clocks to
maintain sub-10ms skew between geographically diverse data
centers. Specifically, Google has placed atomic and GPS clocks in
their data centers. TrueTime runs a daemon that talks to these clocks,
both in its own data center and in others. TrueTime is able to weed
out clocks whose timing information is not reliable (e.g., due to
network latency)~\citep{Corbett2012}.

TrueTime is a proprietary library that requires special hardware, thus
it would not be able to be used directly as a solution to Ceph's
asynchronous replication problem.

%% NOTE: If we are going to talk about this, we need more information
%% However, if a similar open source protocol became available,
%% it could merit deeper analysis. %% TODO This exists, we talked about
                                %% it, it's called Cockroach
