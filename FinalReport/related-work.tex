\chapter{Related Work}
\label{sec:rel-work}

Creating a consistent snapshot of a Ceph cluster is difficult because
Ceph is distributed: The system does not have a single or
small group of nodes that can track every transaction. Instead, the
cluster must aggregate knowledge from potentially thousands of nodes 
to capture a complete point-in-time snapshot of its data. 
Significant research efforts have been directed toward 
taking snapshots as well as synchronizing and ordering events 
in distributed systems. Unfortunately, none of these
approaches can address the particular constraints of Ceph outlined
in Chapter \ref{sec:description} without modification.

\section{Logical Clocks}

Lamport clocks use a ``logical clock'' to establish a partial ordering
of events in distributed systems. If an ordering of the events and
messages between nodes in a distributed system can be established, and
this ordering is complete enough to take into account all
dependencies, a snapshot may be obtained by observing which events
happen before other events. 

Lamport clocks try to order events and
messages by a logical timestamp, a counter that increments following
each event or message pass, rather than by some physical time that
could be affected by clock drift. This ordering method works well if
the system can reason about events that are occurring between its own
nodes. However, this type of partial ordering fails to guarantee
correct ordering in the case of out-of-band communication as described in Chapter \ref{sec:description}; the system
cannot order the events that occur outside of the system as it has no
knowledge of the order of those events. These clocks might still be
useful for reasoning about interactions that occur strictly within the
Ceph cluster, but they require a second system that supports
out-of-band communication to meet Ceph's consistency
guarantee~\citep{Lamport1978}.

Vector clocks are very similar to Lamport clocks. Instead of
maintaining a single logical time, each node stores a vector of
times. Each element of the vector represents other nodes in the system.
The value of each element is the time of the last event received
by that given node in the system. The ordering of
events is based on the relative times stored in the vectors. All
messages contain the sending node's vector and a node
updates its vectors when it receives a message. Vector clocks can be
superior to Lamport clocks for some purposes because they do not assign
an arbitrary counter representing time to events. Instead, they use the relative times of to other nodes to determine the partial order of the events. However, like Lamport clocks, out-of-band communication is not accounted for
and events could be incorrectly ordered~\citep{Fidge1988}.

\section{Network Time Synchronization}

As mentioned in Chapter \ref{sec:description}, total ordering can be
achieved through timestamps. In theory,
this would be able to provide a total ordering of events including
any out-of-band communication. Clock drift and skew complicate this
approach: When multiple clocks are in use in a distributed system,
the clocks must be synchronized in order
to use timestamps to order events across nodes. 

A synchronization algorithm could be effective should it provide
sufficiently tight, provable bounds on the drift of a given clock and
on the skew across the distributed system. These bounds provide a method
to obtain a consistent snapshot as we described in our algorithm in 
Chapter \ref{sec:approach}.

Many clock synchronization algorithms already exist. However, 
these algorithms tend to have major shortcomings when
they are considered for application to this problem within a
data center. A synchronization method must have bounds 
on the clock error in order events in a snapshot, and these 
error bounds must also be a hard upper bound in
order for Ceph to make guarantees about the consistency of its
snapshots. The following subsections describe different time
synchronization methods and protocols.

\subsection{Network Time Protocol}

NTP is a robust algorithm for time
synchronization~\citep{Burbank2010}. NTP is designed to
synchronize time at geographically separated computers over the internet. As a
result, it is resilient to node failure, network unreliability, and
poor clock quality. NTP requires very little information about the
network configuration and nodes on which it is operating to provide useful
synchronization. NTP uses statistical estimators to predict
future clock performance. Notably, NTP does define a maximum error
term in relation to a single root time source. Given this maximum error 
term, NTP is a potential choice as a time synchronization method in 
our algorithm.

A better solution is possible, however. NTP's focus on robustness
causes compromises for the freeze time. NTP consistently overestimates
uncertainty (see Chapter~\ref{sec:results} for an in-depth 
examination of this issue). A protocol designed for local area network
synchronization could likely do away with some of the complexity of
NTP, make more assumptions about network configuration, and as a
result provide tighter bounds on its error.

In Chapter~\ref{sec:results}, we analyze the
performance of an NTP implementation, \texttt{ntpd}, across various clock and
network conditions. The \texttt{ntpd} daemon was chosen for its common use and
ease of access to relevant calculated parameters. \texttt{Chrony} is another
potential choice for an NTP implementation, although it lacks some
diagnostic reporting. %% TODO these were flagged as needing citations
                      %% but I'm not convinced we really need them?

\subsection{Precision Time Protocol}

Precision Time Protocol (PTP) is designed for 
tightly synchronizing computers on a local network
~\citeyearpar{2008}. To gain even better 
synchronization with tighter bounds on clock uncertainty than NTP, 
a user may use specific hardware that supports PTP. Hardware that supports
PTP is generally included in current data center switches already. The type
of hardware that supports PTP helps decrease the amount of random variation in
message latency in a network, enabling extremely accurate measurements of time.
These measurements should in theory provide extremely tight uncertainty
bounds. However, the protocol does not specify (and the implementation
does not include) an upper bound error value. This means that this
protocol is not currently suited for use with our algorithm.

If performance of NTP is found to give unsatisfactory clock uncertainty bounds,
it would be worth considering extending PTP to give an upper bound uncertainty
value (Chapter~\ref{sec:impl} 
has a brief discussion of this). The implementation of a maximum error term 
would be sufficient for PTP to be used with our algorithm. 

% NOTE: Don't think these are needed. Awkward to add implementation details 
%     into related work
%Linux PTP is an implementation of the Precision Time Protocol 
%(PTP)~\citeyearpar{2008}. PTPd is another
%implementation~\citeyearpar{2008}.

\subsection{Wireless Synchronization}

Surprisingly, wireless time synchronization is a much easier
problem. Wireless signal propagation times are easy to model and
as a result time synchronization is easy to perform with a high
level of accuracy. Protocols such as PulseSync take advantage of this
observation to achieve highly synchronized clocks~\citep{Lenzen2010}.

If highly synchronized clocks are a requirement, a wireless
synchronization protocol could be designed that would allow for
a high level of confidence on the clock uncertainty bounds. 
This would require specialized hardware, and one of the goals of 
this project is to provide a solution that can work on any commodity hardware. 

GPS Synchronization is a special case of wireless synchronization that
can provide a nearly perfect time source around the world. It is
advisable to incorporate a GPS time source (or similarly accurate time
source) into the master clock of any implementation of our
algorithm in order to achieve tighter uncertainty bounds, 
though adding this would require specialized hardware.
The details of this are discussed in Chapter~\ref{sec:impl}.

\section{TrueTime}

Google has a number of time-sensitive applications, such as
the synchronously geo-replicated Spanner database. Synchronous
geo-replication of database transactions requires very strong time
guarantees because database transaction ordering is important for 
reliability and security. 

A library called TrueTime was developed to support these applications.
TrueTime does not provide current time in the way that standard time
libraries do, such as for NTP or PTP. Instead, the ``current time'' it gives is a range that is guaranteed to contain
the real current time. The bounds are claimed to be generally less than 10
milliseconds. This range in the real current time allows for rapid 
throughput on the database while still maintaining definitive transaction ordering. When TrueTime's guaranteed bounds start to spread, 
it will throttle writes to maintain that ordering~\citep{Corbett2012}.

TrueTime relies on having extremely accurate and precise clocks 
to maintain sub-10ms skew between geographically diverse data centers. Specifically, Google has placed atomic and GPS clocks in their data 
centers. TrueTime runs a daemon that talks to these clocks, both in its own data center and in others. TrueTime is able to weed out clocks whose 
timing information is not reliable 
(e.g. due to network latency)~\citep{Corbett2012}.

TrueTime is a proprietary
library that requires special hardware, thus it would not be able to
be used directly as a solution to Ceph's asynchronous replication
problem. 

%% NOTE: If we are going to talk about this, we need more information
%% However, if a similar open source protocol became available,
%% it could merit deeper analysis. %% TODO This exists, we talked about
                                %% it, it's called Cockroach
