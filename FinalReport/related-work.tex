\chapter{Related Work}
\label{sec:rel-work}

Creating a consistent snapshot of a Ceph cluster is difficult because
Ceph is inherently distributed. The system does not have a single or
small group of nodes that can track every transaction. Instead, the
cluster must aggregate knowledge from thousands of nodes to capture a
complete point-in-time snapshot of its data. Significant research
efforts have been directed toward taking snapshots, and synchronizing
and ordering events in distributed systems. Unfortunately, none of he
approaches we reviewed can address the particular needs of Ceph
without modification.

\section{Logical Clocks}

Lamport clocks use a ``logical clock'' to establish a partial ordering
of events in distributed systems. If an ordering of the events and
messages between nodes in a distributed system can be established, and
this ordering is complete enough to take into account all
dependencies, a snapshot may be obtained by observing which events
happen before other events. Lamport clocks try to order events and
messages by a logical timestamp, a counter that increments following
each event or message pass, rather than by some physical time that
could be affected by clock drift. This ordering method works well if
the system can reason about events that are occurring between its own
nodes. However, this type of partial ordering fails to guarantee
correct ordering in the case of out-of-band communication; the system
cannot order the events that occur outside of the system as it has no
knowledge of the order of those events. These clocks might still be
useful for reasoning about interactions that occur strictly within the
Ceph cluster, but they require a second system that supports
out-of-band communication to meet our consistency
guarantee~\citep{lamport}.

Vector clocks are very similar to Lamport clocks. Instead of
maintaining a single logical time, each node stores a vector of
times. Each element of the vector represents the last time known to
have been seen on a given other node in the system. The ordering of
events is based on the relative times stored in the vectors. All
messages contain the sending process's vector and a process
updates its vectors when it receives a message. Vector clocks can be
superior to Lamport clocks for some purposes because they do not assign
an arbitrary time to events. Instead, they use the relative times of
their processes to determine the partial order of the events. However,
like Lamport clocks, out-of-band communication is not accounted for
and events could be incorrectly ordered~\citep{vector}.

\section{Network Time Synchronization}

We could also try to achieve ordering through timestamps. In theory,
this would be able to provide a total ordering of events independent
of any out-of-band communication. Clock drift and skew complicate this
approach: When multiple time sources are in use in a distributed system,
the clocks must be synchronized to a high degree of accuracy in order
to use timestamps to order events across nodes. In a real distributed
system, generally each machine is just using its own RTC
circuit. These are reasonably good for ordinary use, but do not have
the reliability of precision we need. %% TODO? cite intel spec sheet

A naive approach to a distributed snapshot algorithm uses timestamps
alone to establish a total ordering of events in a system. This works
well in systems with centralized controllers or logs that have
knowledge of all events. In a system with highly distributed control
like Ceph, each node only sees a small fraction of events. Each node
contains its own clock, and these clocks tend to drift perceptibly
over time. Clocks across a distributed system become skewed as they
drift out of sync. They must be synchronized regularly to combat this
effect.

Consider a cluster containing multiple nodes. Within this cluster,
consider event $i$ and event $j$ such that $j$ depends on $i$. Without
some method of ensuring bounded clock drifts, it is impossible to
guarantee that a given timestamp at a specific node is correct with
reference to the entire system. One node (with a slightly fast clock)
may claim that event $i$ occurred at a point in time later than the
event happened as seen by some ``objective'' observer. Event $j$,
processed by a node with a slightly slow clock, could then ``occur''
(be timestamped) before event $i$. A snapshot algorithm might now
believe it is acceptable to include event $j$ and not event $i$. Such
a snapshot would be inconsistent.

A synchronization algorithm could be effective should it provide
sufficiently tight, provable bounds on the drift of a given clock and
on the skew across the distributed system. These bounds would allow
the file system to gain a clear understanding of what knowledge of
event ordering it has when taking a snapshot.

Many clock synchronization algorithms already exist and are in wide
use. However, these algorithms tend to have major shortcomings when
they are considered for application to this problem in an
arbitrary-scale data center. A synchronization method must have
provable error bounds and reasonable communication complexity for
timestamps alone to order events in a snapshot. We must also be able
to prove the reliable correctness of the error bound calculations in
order for Ceph to make guarantees about the consistency of its
snapshots.

\subsection{Network Time Protocol}

NTP is a robust algorithm for time
synchronization~\citep{ntp}. Currently it is among the most
widely-deployed time synchronization algorithms. NTP is designed to
synchronize geographically separated computers over the internet. As a
result, it is resilient to node failure, network unreliability, and
poor clock quality. NTP requires very little information about the
network and nodes on which it is operating to provide useful
synchronization. It uses a number of statistical estimators to predict
future clock performance. Notably, NTP does define a maximum error
term in relation to a single root time source. As a result, NTP is a
potential choice for a network time protocol in our algorithm.

A better solution is possible, however. NTP’s focus on robustness
causes compromises for the freeze time. NTP consistently overestimates
uncertainty (see figure~\ref{fig:safety-data} and
chapter~\ref{sec:analysis} more generally for an in-depth examination of
this issue). A protocol designed for local area network
synchronization could likely do away with some of the complexity of
NTP, make more assumptions about network configuration, and as a
result provide tighter bounds on its error.

In chapters~\ref{sec:results} and~\ref{sec:analysis}, we analyze the
performance of an NTP implementation, ntpd, across various clock and
network conditions. The ntpd daemon was chosen for it common use and
ease of access to relevant calculated parameters. Chrony is another
potential choice for an NTP implementation, although it lacks some
diagnostic reporting. %% TODO these were flagged as needing citations
                      %% but I'm not convinced we really need them?

\subsection{Precision Time Protocol}

PTP is designed for tightly synchronizing computers on a local network
<TODO citation>. To gain even better synchronization, a user may use
specific hardware that supports PTP. This hardware is generally
included in current data center switches already. This hardware helps
PTP decrease the amount of random variation in message latency in a
network, enabling extremely accurate measurements of time. These
measurements should in theory provide extremely tight uncertainty
bounds. However, the protocol does not specify (and the implementation
does not include) an upper bound error value. This means that this
protocol is not currently suited for use with our algorithm.

If performance of NTP is found to be unsatisfactory, it would be worth
considering extending PTP (chapter~\ref{sec:impl} has a brief discussion
of this). The implementation of a maximum error term would be
sufficient for PTP to be supported. Linux PTP is an implementation of
the Precision Time Protocol (PTP) <TODO citation>. PTPd is another
implementation <TODO citation>.

\subsection{Wireless Synchronization}

Surprisingly, wireless time synchronization is a much easier
problem. Wireless signal propagation times are very easy to model and
as a result time synchronization is easy to perform with a very high
level of accuracy. Protocols such as PulseSync take advantage of this
observation~\citep{lenzen_optimal_2009}.

If extremely small freeze windows are a requirement, a wireless
synchronization protocol could easily be designed that would allow for
a very high level of confidence. This would require significant
specialized hardware, and one of the goals of this project is to
provide a solution that can work on commodity hardware. 

GPS Synchronization is a special case of wireless synchronization that
can provide a nearly perfect time source around the world. It is
advisable to incorporate a GPS time source (or similarly accurate time
source) into the master clock of any implementation of our
algorithm. The details of this are also discussed in
chapter~\ref{sec:impl}.

\section{TrueTime}

Google has a number of very time-sensitive applications, most notably
the synchronously georeplicated Spanner database. Synchronous
georeplication of database transactions requires very strong time
guarantees because database transaction ordering is important. A
library called TrueTime was developed to support these applications.

\subsection{TrueTime.now()}

TrueTime does not provide current time in the way that most time
libraries do. Instead, it gives a range that is guaranteed to contain
the current time. The bounds are claimed to be generally less than 10
milliseconds. This allows for very rapid throughput on the database
while still maintaining definitive transaction ordering. When
TrueTime’s guaranteed bounds start to spread, it will throttle writes
to maintain that ordering~\citep{spanner}.

\subsection{TrueTime's Clocks}

TrueTime relies on having extremely good clocks to maintain sub-10ms
skew between geographically diverse data centers. Specifically, Google
has placed atomic and GPS clocks in their data centers. TrueTime runs
a daemon that talks to these clocks, both in its own data center and
in others. A variation on Marzullo’s algorithm is used to weed out
clocks whose timing information is not reliable (e.g. due to network
latency)~\citep{spanner}.

\subsection{Applicability}

As a concept, TrueTime is interesting. However, as it is a proprietary
library and as it requires special hardware, it would not be able to
be used directly as a solution to Ceph’s asynchronous replication
problem. However, if a similar open source protocol became available,
it could merit deeper analysis. %% TODO This exists, we talked about
                                %% it, it's called Cockroach
