\chapter{Commonly Used Terms}
\label{sec:terms}

\section{General terms}
\begin{itemize}
  \item Cluster: computers that are networked together to perform a
    common task. A Ceph cluster is a bunch of computers that store
    data using Ceph.
    
  \item Node: a single computer in a cluster.

  \item Snapshot: a consistent recording of the entire state of a Ceph cluster

  \item Consistency: the partial ordering of events to preserve the
    causal relationships between them.

  \item Node Freeze: a hold on processing all writes that are sent to
    that node.

  \item Clock Drift: the natural drift over time that a computer
    clock experiences.

  \item Time Synchronization Protocol: a protocol that the nodes in
    a cluster implement to synchronize their clocks as they drift.

  \item Real Time: the true, global time of the cluster.

  \item Node Time: the time that the clock on a particular node reads;
    not guaranteed to be the same as real time.

  \item Out-of-Band Communication: communication between users of a
    Ceph cluster that Ceph has no way of detecting or recording.

  \item Node Failure: when a node in the cluster unexpectedly
    crashes, either temporarily or permanently.
\end{itemize}

\section{Ceph specific terms}

\begin{itemize}
  \item RADOS (Reliable Autonomic Distributed Object Store): Object
    store service backing the Ceph file system.
  
  \item OSD (Object Storage Device): A given storage node within
    RADOS.

  \item PG (Placement Group): Aggregation of object groups for
    tracking object placement on OSDs within RADOS

\end{itemize}
