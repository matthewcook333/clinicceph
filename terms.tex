\chapter{Commonly Used Terms}
\label{sec:terms}

\section{General terms}
\begin{itemize}
  \item Cluster: Computers that are networked together to perform a
    common task. A Ceph cluster is a single deployment of the Ceph
    file system across a group of computers. 
    
  \item Node: A single computer in a cluster.

  \item Snapshot: A consistent recording of the entire state of a Ceph cluster. 
  This includes the state of files and objects stored within the cluster.

  \item Consistency: The partial ordering of events to preserve the
    causal relationships between them.
    
  \item Strong Consistency: Ceph is strongly consistent. It requires 
  that ordering of events be maintained even in the face of out-of-band
  communication. It also requires that data stored in the cluster will
  be returned correctly when asked for or be unavailable. It does not 
  allow data to be returned corrupt.

  \item Node Freeze: A hold on processing all writes that are sent to
    that node. Reads may continue to be processed.

  \item Clock Drift: The natural drift over time that a computer
    clock experiences.

  \item Time Synchronization Protocol: A protocol that the nodes in
    a cluster implement to synchronize their clocks as they drift.

  \item Real Time: The true, global time of the cluster. This can be considered
  the time advertised by a standards organization or a perfect clock.

  \item Node Time: The time that the clock on a particular node reads;
    it is not guaranteed to be the same as real time, although we hope it is
    close.

  \item Out-of-Band Communication: Communication between users of a
    Ceph cluster that Ceph has no way of detecting or recording.

  \item Node Failure: When a node in the cluster
    becomes unavailable, either temporarily or permanently.
\end{itemize}

\section{Ceph specific terms}

\begin{itemize}
  \item Reliable Autonomic Distributed Object Store (RADOS): The core services and software backing the Ceph file system.
  
  \item Object Storage Device (OSD): A single storage unit. Functionally equivalent to a single computer or node.

  \item Placement Group (PG): Objects stored in a Ceph cluster are aggregated into
   Placement Groups by a distributed hash function. Each Object Storage Device
   contains a large number of Placement Groups.

\end{itemize}
